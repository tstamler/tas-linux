\section{Implementation}\label{sec:impl}

We have implemented \taas from scratch in 10,127 lines of C code
(LoC). \taas' fast path comprises 2,931 LoC. The slow path comprises
3,744 LoC. The user-level library providing the POSIX sockets API is
dynamically linked to unmodified application binaries and comprises
3,452 LoC.

\paragraph{Fast path.}
The fast path runs in a user-level process, separate from the
applications. The fast path uses DPDK~\cite{dpdk} to directly access
the machine's NIC, bypassing the Linux kernel. Unlike systems that
rely on batching to reduce kernel-user switches, \taas uses a
configurable number of dedicated host cores, which we can vary based
on the offered network load. Each core replicates a linear packet
processing pipeline and exposes a queue pair to the slow path and to
each application context to avoid synchronization. The NIC's RSS
mechanism ensures that packets within flows are assigned to the same
pipeline and not reordered.

\paragraph{Slow path.} The slow path runs as a separate thread within
the fast process. To bootstrap context queues, we require applications
to first connect to the slow path via a named UNIX domain
socket. Applications use the socket to set up a shared memory region
for the context queues. The slow path also uses the socket for
automatic cleanup, to detect when application processes exit by
receiving a hangup signal via the corresponding socket.

\subsection{Limitations}

\paragraph{Fixed connection buffer sizes.}
\softtcp requires connection send and receive buffers to be fixed upon
connection creation. We do not currently implement any buffer resizing
depending on load.  For workloads with large numbers of inactive
connections, buffer resizing (via additional management commands) is
desirable.

\paragraph{TCP slow start.} Our prototype does not fully implement the
TCP slow start algorithm. Instead, we current double the sending rate
every RTT until we reach steady-state. Since our measurements are only
concerned with steady-state performance, this limitation does not
impact the reported results.
