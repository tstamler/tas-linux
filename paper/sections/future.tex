\section{Future Work}\label{Future}

Our design and implementation of Linux-Integrated TAS can handle slow path
operations such as connection setup/teardown as well as ARP packets. However,
there is additional slow path functionality that we leave to future work.

\subsection{Congestion Control}
The slow path can take advantage of Linux's policies for congestion control
and forward this information to the fast path for enforcement. This would entail
forwarding packets with ECN bits set between Linux and the slow path. The slow
path would then observe Linux's response to the network congestion and mimic
it. In addition, we may want to forward ECN packets received from the network to
Linux or forcibly generate and send it ECN packets to get it to more aggresively
back off the network. 

\subsection{Connection Timeouts and Resets}

We can utilize FIN and RST packets sent by Linux for various reasons to 
potentially shutdown remote connections. In order to do this intelligently, we'd
need to identify the different reasons why Linux would want to close a 
connection and selectively choose the ones we want to fulfill. In order to 
prevent regular timeouts, the slow path would need to occasionally send packets 
to Linux to allow Linux to keep its connections from timing out, which we 
currently do not do. 

\subsection{OpenVSwitch Integration}

OpenVSwitch\cite{openvswitch} (OVS) is a virtual switch commonly used in 
datacenters for reprogrammable traffic steering at very large scale. It supports
a number of useful traffic identification and modification options at minimal 
overhead. 

While our implementation currently interfaces with Linux, it would be even more
useful to be able to work alongside OVS. Luckily, OVS supports Linux virtual 
interfaces like tap and tun devices, so integration should be fairly simple. 
However, because tap devices operate at the Ethernet layer in the network stack,
our current implementation would only support OVS functions that filter or 
switch based on MAC addresses. This may be mildly useful, but in order to get 
all of the functionality of OVS, we'd want to use a tun device instead. By doing
this, we would have a IP layer virtual network interface, and OVS can route and 
filter based on IP addresses instead.
