\section{Introduction}\label{Introduction}

Networking speeds have become faster while CPUs have not, causing network packet
processing efficiency to become important for datacenter networks. Datacenter
applications continue to want high throughput and low latency access to the
network along with the guarantees provided by TCP: lossless in-order delivery of
packets, but this comes at the cost of consuming an increasing fraction of CPU
processing resources. For example, nearly 70\% of packet processing time for a
simple echo server application is spent in the Linux networking stack 
\cite{peter:arrakis}.

To cope with this, many alternative TCP stacks have been proposed that seek to
increase the efficiency of packet processing. TAS (TCP Acceleration as a
Service) splits TCP packet processing into a fast path and a slow path. The fast
path handles common data path operations such as handling in-order delivery of
packets from established connections and generating acknowledgements. The slow
path handles less common, control path operations such as connection management,
congestion control, and connection timeouts. Both the fast path and slow path
operate as user-level processes.

Implementing a TCP stack in userspace comes with a few drawbacks. Namely, the
Linux TCP stack contains a lot of functionality and information about the
network that is hard to replicate in userspace. Ideally, a userspace networking 
stack should make the same decisions about connection management, security,
congestion, etc. as the Linux stack.

We present Linux-Integrated TCP Acceleration as a Service, an extension to TAS
that interfaces with Linux for some slow path operations. The slow path now
sends some packets to Linux, observes Linux's response, and mimics it. In this
way, TAS can gain some of the information and functionality of the Linux TCP
stack, such as firewall and network information (e.g., ARP tables), while
retaining the performance of fast path operations.

Our paper makes the following contributions:

\begin{itemize}
\item We design and implement a method for the TAS to interface with Linux for
  some slow path operations (connection setup and teardown, ARP) in order to
  gain information and functionality.

\item We evaluate our implementation and show that we introduce no overheads
  for fast path operations. Connection setup and ARP slow down significantly,
  but these operations are uncommon enough that they do not affect the
  throughput seen by the application.
\end{itemize}

In the remainder of our paper, we provide some background on TAS and virtual
network devices in Section \ref{Background}. We discuss the design and
implementation of Linux-integrated TAS in Section \ref{Design}. We evaluate
our implementation in Section \ref{Eval} and finally conclude and discuss
future work in Section \ref{Conclusion}.