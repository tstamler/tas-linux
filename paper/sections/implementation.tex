\section{Implementation}

In order to achieve a Linux-integrated TAS implementation, the following 
changes were implemented and functionalities added.

\subsection{Tap Initialization and Interaction}

At the time of TAS's initialization, a tap device is created with Linux.
We control the name and IP of this tap device, but Linux and anyone using the
operating system with root privileges can configure it to do whatever they
want, and TAS will reflect these changes. At initialization, we need to
save the MAC address that is assigned to the tap device so that future packets
written to the tap device can copy and use it. 

Writing packets to the tap device is simple and can be done from anywhere in the
slow path. It is merely a POSIX \textit{write} function call with the buffer and a length, but we 
make sure to copy the saved MAC address for the tap device over the desination
MAC field in any packets being written to the tap device. 

Reading packets from the tap device is done in a separate thread that can block
with \textit{epoll} while waiting for packets to be received from Linux. Once packets are
received, they are parsed and handled according to headers, flags, and existing
connection state. With this parsing already done, it enables us to more easily
handle more types of Linux packets and add more functionality. 

\subsection{Tap Control Packets and Network Interface}

To more easily create and format packets to write to the tap device, a new
interface was written based on the existing TAS control packet creation
interface, but reversed. When a control packet such as an \textit{ACK} needs to be 
generated, connection state is looked up and crafted into a packet, with
TAS being the source and the remote host being the destination. Instead,
with our tap control interface, a packet is crafted with the remote host as
the source and Linux as the destination. This reversal applies to IP addresses,
MAC addresses, sequence and acknoledgement numbers, and TCP timestamp options.

In addition to a new interface for generating packets for the tap device, a new
interface was needed for forwarding raw packets from the tap device to the 
network. By default, TAS only has the existing control interface for 
creating and sending certain packets to the network from the slow path, but 
not an interface for sending raw buffers directly to the network. We added this
functionality so that we can forward packets directly from Linux to the remote
host with minimal modifications.

\subsection{Modified Socket Library}

Because Linux does not just acknowledge arbitrary \textit{SYN} packets, we had to do some
extra work at the library level in order to prepare Linux. Typically, the 
application would interact with Linux through the socket library, but with 
TAS, all socket operations are intercepted and forwarded instead of being
delivered to Linux. In this environment, Linux is completely unaware of the 
application listening for connections. To rectify this, we duplicate all 
necessary socket operations through libc. When the application calls a socket 
operation that we need, the interposition layer on top of libc will call both 
TAS and libc so that Linux is aware of all the same information that 
TAS is. The functions we currently handle for this are socket, bind, 
fcntl, setsockopt, listen, and accept. 

To ensure proper bookkeeping of the different file descriptors, we maintain a map
of TAS file descriptors to Linux file descriptors and only return TAS
descriptors to the application. Whenever one of the aforementioned functions is 
called, the TAS file descriptor provided by the application is 
cross-referenced with the map in order to find the proper file descriptor to use
with libc. 

\subsection{ARP Support}

In our implementation of TAS, Linux cannot interact directly with remote 
hosts, and ARP requests and replies are no longer handled after connection state
is initialized. Because Linux will likely need to do an ARP request to obtain 
the remote host's MAC address, and because we don't want to make any 
modifications to the fast path, we handle ARP requests in the slow path.

ARP replies are generally sent by the host to which the corresponding address
belongs, but in order to provide Linux the information it needs we fake these 
replies inside the slow path. Upon receiving an ARP request from Linux, we 
search through TAS's ARP table to see if the address is there. If it is, 
we generate a reply and write it back to Linux. If it is not, we forward the 
packet to the network. Because TAS was unaware of this address, we know
it does not maintain any connection state for it, and the corresponding ARP 
reply should be forwarded to the slow path, where it can be written to the tap
device. 

\subsection{Connection Setup}

Finally, we will outline the general changes we had to make to TAS to get
it to cooperate with Linux during connection setup. Once the \textit{SYN} packet arrives,
 it will naturally be forwarded to the slow path for
connection establishment. After allowing TAS to do its normal connection 
state initialization, we write the \textit{SYN} packet to Linux. At this point, TAS
would generate an event for the \textit{SYN/ACK} to be generated asynchronously. We remove
this funcationality and instead wait for Linux to send a \textit{SYN/ACK}. Linux will 
first do an ARP request, which we handle, followed by a \textit{SYN/ACK}. We forward this
packet to the network, then follow the normal TAS flow of updating the 
connection state and signalling the application, as well as the additional work 
of generating an \textit{ACK} packet to send to Linux. 

